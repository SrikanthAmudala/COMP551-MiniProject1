\documentclass[conference]{IEEEtran}
\IEEEoverridecommandlockouts
% The preceding line is only needed to identify funding in the first footnote. If that is unneeded, please comment it out.
\usepackage{cite}
\usepackage{amsmath,amssymb,amsfonts}
\usepackage{algorithmic}
\usepackage{graphicx}
\usepackage{comment}
\usepackage{textcomp}
\usepackage{lineno,hyperref}
\usepackage{caption}
\usepackage{subcaption} 
\usepackage{tabularx}
\usepackage{colortbl}
\usepackage{hhline}
\usepackage{url}
\usepackage{csquotes}
\def\BibTeX{{\rm B\kern-.05em{\sc i\kern-.025em b}\kern-.08em
    T\kern-.1667em\lower.7ex\hbox{E}\kern-.125emX}}
\begin{document}

\title{A Comparison of Logistic Regression Vs. LDA for Wine Quality Detection and Breast Cancer Detection}

\author{
	\IEEEauthorblockN{
		Julien Verecken\IEEEauthorrefmark{1},
		Srikanth Amudala\IEEEauthorrefmark{2} and
		Kamal Maanicshah\IEEEauthorrefmark{2}
		 }
	\IEEEauthorblockA{
		\IEEEauthorrefmark{1}address1\\
		Email: }
	\IEEEauthorblockA{
		\IEEEauthorrefmark{2} Concordia Institute of Information Systems Engineering,\\ Concordia University,1455 Boulevard de Maisonneuve O, Montréal, QC H3G 1M8\\
	    Email:srikanth.amudala@mail.mcgill.ca; kamal.mathinhenry@mail.mcgill.ca}
	}

\maketitle
\begin{abstract}
	The write-up discusses the implementation and performances of logistic regression and LDA models based on the results obtained by applying the models on a wine quality data set and a breast cancer data set. Based on statistical analysis we perform feature extraction instead of using all the features.  
\end{abstract}
\section{Introduction}
In this project we first implement Logistic regression and then use a Linear Discriminant Analysis model for classification tasks. Given a data set containing $n$ samples, $X = {x_1, x_2,...,x_n}$ with each sample $X_i = {x_{i1},x_{i2},...,x_{im}}$ where $m$ is the number of features.
\section{Dataset}
\subsection{Red wine quality}
The models used in this project will be trained to achieve a binary classification on the ``redwine'', previewed in Fig. \ref{redwine_head}.
% preview image
The ``quality'' target ranges from 0 to 10. The thresholding operation described in \eqref{thresholding} maps the multiclass target to a binary value. The class ``1'' will be further referenced as the ``good'' wine class and similarly the other as ``bad''.
\begin{align}
	y_{\text{binary}} = y_{\text{multiclass}} > 5
\end{align}
The histograms specific to each feature, separated by class (Fig. \ref{redwine_hist}) allows to perform a first analysis.




\section{Feature Extraction}
\section{Results}
\section{Discussion}



\bibliographystyle{ieeetran}
\end{document}
